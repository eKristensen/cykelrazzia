MAIL EMNE, SUBJECT: Cykelrazzia 24. maj - IMPORTANT: Bike Raid May 24th

*** ENGLISH BELOW ***

Til beboerne på P. O. Pedersen Kollegiet,

Trafikudvalget varsler hermed en cykelrazzia torsdag d. 24. maj 2018.


PLADSPROBLEMER I CYKELSKURENE

Som flere af jer formentlig har oplevet, kan det indimellem være svært at finde en plads til sin cykel i kollegiets cykelskure, så man i stedet må parkere sin cykel op ad skuret udenfor.

Problemet har flere årsager, bl.a. følgende:
* Cykler stilles dårligt på plads, så de fylder unødvendigt meget.
* Skurene rummer cykler der ikke er brugbare, eller simpelthen ikke tilhører kollegiets beboere.
* Parkerede cykler skal være brugbare (se Husordenens § 64, Stk. 2), for at skurene ikke bliver genbrugsstationer med ødelagte/halve cykler.

Det skal understreges at beboernes gæster gerne må have deres cykler stående i skuret, men naturligvis ikke må lade cyklerne stå i skuret i månedsvis.


CYKELRAZZIA

Trafikudvalget står for oprydning på kollegiets udendørsarealer, primært i cykelskurene, og løser dermed det sidstnævnte problem. Cykelrazziaen udføres efter Husordenens § 61.

Cykelrazziaen er i dag (torsdag d. 12. april) varslet, efter Husordenens § 65, Stk. 2, med opslag på rådstavlen og på alle køkkendøre samt via denne mail.

Dette års cykelrazzia gennemføres næsten ligesom sidste år som følger:

1) En uge før cykelrazziaen fastgøres der en grøn tråd til cykelstyrene på alle cykler der befinder sig i og ved cykelskurene
2) Det er nu den enkelte cykels ejers ansvar at fjerne denne grønne tråd inden cykelrazziaen som indikation på at cyklen er i brug.
3) På dagen for cykelrazziaen vil alle cykler der stadig har en tråd fastgjort blive fjernet af Trafikudvalget.

Alle cykler der stadig har en strip fastgjort når razziaen udføres, vil blive flyttet op til det store skur for enden af parkeringspladsen, hvor de kædes og låses sammen. Herefter kan man kun ved henvendelse til inspektørkontoret få sin cykel igen (Trafikudvalget står kun for razziaen, og har ikke nøgle til låsene). To uger efter razziaen afhenter politiet de tilbageværende fjernede cykler, primært for at undersøge om nogen af dem er stjålne, og der er dermed ikke mulighed for at få en gratis cykel eller gratis reservedele fra de fjernede cykler.


ÉN CYKEL PER BEBOER

Kollegiets beboere må kun hver have én cykel stående på cykelarealerne (jf. Husordenens § 63). Kun i særlige tilfælde kan der gives dispensation fra dette krav. For at frigøre plads i cykelskurene bedes gamle cykler derfor altid fjernes af ejeren når de ikke længere bruges, ved fraflytning eller lignende. Der er for alle parter uholdbart at vente på at cyklen bliver fjernet af andre.


SÆRLIGE OPLYSNINGER

Naboer: Alle opfordres til at minde deres naboer om cykelrazziaen, særligt hvis der er nogen der ikke læser deres post, kollegie-e-mails og køkkendørssedler. Har man en nabo der er bortrejst fra nu af og til efter cykelrazziaen, kan det være nødvendigt at man fjerner strips'en for vedkommende.
Total sikkerhed: Hvis du er meget nervøs for cykelrazziaen, står det dig frit for at opbevare din cykel indendørs den ene dag det står på.
Det aflåste rum: Der foretages ikke cykelrazzia i det aflåste rum i skuret ud for blok 2.

Venlig hilsen
Trafikudvalget (Køkken O)

Kontakt: raadet@pop.k-net.dk


---------------------------------
ENGLISH

To the residents of P. O. Pedersen Kollegiet,

The Traffic Committee hereby announce a bike raid on Thursday May 24 2018.


LACK OF ROOM IN THE BIKE SHEDS

As several of you may have experienced, it is sometimes difficult to find room for your bike in the dormitory's bike sheds, meaning that you have to park your bike up against the shed on the outside instead.

The problem has multiple causes, among others the following:
* Bikes are poorly placed, so they take up too much room.
* The sheds contain bikes that are not usable, or simply don’t belong to the dormitory’s residents.
* Parked bikes must be usable (refer to § 64, Section 2 of the House Rules), to avoid a situation where the sheds become recycling centres with broken/half bikes.

It is stressed that the residents' guests are allowed to keep their bikes in the shed, provided, of course, that they do not store them there for months.


BIKE RAID

The Traffic Committee is responsible for clean-up on the dormitory's outdoor areas, primarily in the bike sheds, and thereby solve the last-named problem. The bike raid is carried out according to § 61 of the House Rules.

The bike raid was announced today (Thursday April 12), according to § 65, Section 2 of the House Rules, with notices on the council board and on all kitchen doors plus in this e-mail.

This year’s bike raid is similar to last year conducted as follows:

1) One week before the bike raid, green threads are attached to the handlebars of all bikes in and nearby the bike sheds.
2) It is then the responsibility of the owner of each bike to remove this thread before the bike raid as an indication that the bike is in use.
3) On the day of the raid all bikes still marked by threads will be removed by the Traffic Committee.

All bikes that still have the strip attached to their handlebars when the raid is conducted, will be moved to the big shed at the end of the parking lot, where they are chained and locked. Hereafter, you can only get your bike back by contacting the inspector's office (the Traffic Committee is only responsible for the raid, and does not have a key for the locks). Two weeks after the raid the police picks up the remaining moved bikes, primarily to investigate if any of them are stolen, and there is thus no possibility of getting a free bike or free spare parts from the moved bikes.


ONE BIKE PER RESIDENT

The residents of the dormitory may only have one bike stored at the bike areas (see the House Rules section 63). Only under special circumstances can the Traffic Committee grant a dispensation from this rule. In order to free space in the bike sheds, old bikes should always be removed by the owner when they no longer are in use, when moving out etc. It is unsustainable for all parties involved just to wait for the bike to be removed by others.


SPECIAL INFORMATION

Neighbors: Everyone is asked to remind their neighbors about the bike raid, especially if there are some who do not read their mail, dormitory e-mails and kitchen door notes. If you have a neighbor who has gone away from now on until after the bike raid, it may be necessary for you to remove the strip for him/her.
Total safety: If you are very nervous about the bike raid, you are free to just keep your bike indoors on the one day of the raid.
The locked room: The bike raid does not cover the locked room in the shed outside block 2.
 

Best regards,
The Traffic Committee (Kitchen O)

Contact: raadet@pop.k-net.dk